\documentclass[]{elsarticle} %review=doublespace preprint=single 5p=2 column
%%% Begin My package additions %%%%%%%%%%%%%%%%%%%
\usepackage[hyphens]{url}

  \journal{N/A} % Sets Journal name


\usepackage{lineno} % add
\providecommand{\tightlist}{%
  \setlength{\itemsep}{0pt}\setlength{\parskip}{0pt}}

\usepackage{graphicx}
%%%%%%%%%%%%%%%% end my additions to header

\usepackage[T1]{fontenc}
\usepackage{lmodern}
\usepackage{amssymb,amsmath}
\usepackage{ifxetex,ifluatex}
\usepackage{fixltx2e} % provides \textsubscript
% use upquote if available, for straight quotes in verbatim environments
\IfFileExists{upquote.sty}{\usepackage{upquote}}{}
\ifnum 0\ifxetex 1\fi\ifluatex 1\fi=0 % if pdftex
  \usepackage[utf8]{inputenc}
\else % if luatex or xelatex
  \usepackage{fontspec}
  \ifxetex
    \usepackage{xltxtra,xunicode}
  \fi
  \defaultfontfeatures{Mapping=tex-text,Scale=MatchLowercase}
  \newcommand{\euro}{€}
\fi
% use microtype if available
\IfFileExists{microtype.sty}{\usepackage{microtype}}{}
\bibliographystyle{elsarticle-harv}
\ifxetex
  \usepackage[setpagesize=false, % page size defined by xetex
              unicode=false, % unicode breaks when used with xetex
              xetex]{hyperref}
\else
  \usepackage[unicode=true]{hyperref}
\fi
\hypersetup{breaklinks=true,
            bookmarks=true,
            pdfauthor={},
            pdftitle={ASSOCIATION BETWEEN WORKING HOURS AND TOTAL BLOOD CHOLESTEROL AMONG ADULTS IN THE UNTIED STATES},
            colorlinks=true,
            urlcolor=blue,
            linkcolor=magenta,
            pdfborder={0 0 0}}
\urlstyle{same}  % don't use monospace font for urls

\setcounter{secnumdepth}{0}
% Pandoc toggle for numbering sections (defaults to be off)
\setcounter{secnumdepth}{0}

% Pandoc citation processing
\newlength{\cslhangindent}
\setlength{\cslhangindent}{1.5em}
\newlength{\csllabelwidth}
\setlength{\csllabelwidth}{3em}
% for Pandoc 2.8 to 2.10.1
\newenvironment{cslreferences}%
  {}%
  {\par}
% For Pandoc 2.11+
\newenvironment{CSLReferences}[2] % #1 hanging-ident, #2 entry spacing
 {% don't indent paragraphs
  \setlength{\parindent}{0pt}
  % turn on hanging indent if param 1 is 1
  \ifodd #1 \everypar{\setlength{\hangindent}{\cslhangindent}}\ignorespaces\fi
  % set entry spacing
  \ifnum #2 > 0
  \setlength{\parskip}{#2\baselineskip}
  \fi
 }%
 {}
\usepackage{calc}
\newcommand{\CSLBlock}[1]{#1\hfill\break}
\newcommand{\CSLLeftMargin}[1]{\parbox[t]{\csllabelwidth}{#1}}
\newcommand{\CSLRightInline}[1]{\parbox[t]{\linewidth - \csllabelwidth}{#1}\break}
\newcommand{\CSLIndent}[1]{\hspace{\cslhangindent}#1}

% Pandoc header
\usepackage{setspace}\doublespacing



\begin{document}
\begin{frontmatter}

  \title{ASSOCIATION BETWEEN WORKING HOURS AND TOTAL BLOOD CHOLESTEROL
AMONG ADULTS IN THE UNTIED STATES}
    \author[San Diego State University School of Public
Health]{Christian Hicks}
   \ead{chicks5799@sdsu.edu} 
      
  \begin{abstract}
  Background. During 2015-2018 in the United States, mean total blood
  cholesterol was over 190 mg/dL. This is closer to being considered
  clinically high at 200 mg/dL than it was to achieving the Healthy
  People goal for 2020 at 177 mg/dL. Previous studies have described
  associations between long working hours and undesirable cholesterol
  health in certain sub-populations. International studies have had
  consistent results within their respective general populations, but
  published research on the general U.S. population is scarce. This
  study sought to describe the effect of working hours on cholesterol
  health amongst the U.S. population at large. Methods. Weighted results
  from the National Health and Nutrition Examination Survey were
  analyzed for this study. Of the original 9,254 participants, 1,715
  were selected by their age, employment status, and relevant health
  factors to be included for the analysis. Multiple linear regression
  was used to measure the effect and significance of working hours on
  cholesterol after adjustment for confounding variables. Results. The
  prevalence of clinically high cholesterol was 37\%, and more than 47\%
  had worked more than 40 hours in the past week. Mean total cholesterol
  was 190.0 mg/dL (SE = 1.4). After adjustment in the linear regression
  model, each hour of work had a weak association with a reduction in
  total cholesterol by 0.28 mg/dL. The largest contributors to higher
  total cholesterol were age and BMI. Observations who worked less than
  full-time had 4.6\% higher prevalence of clinically high total
  cholesterol, though this difference was not statistically significant.
  Discussion. This study does not provide enough evidence that longer
  working hours is associated with declining cholesterol health. Results
  may be confounded by a healthy worker bias; therefore, further studies
  should obtain data from persons who are no longer working because
  health condition could influence employment status. Conclusion.
  Further research is needed to control for the healthy worker effect
  and confirm that the results of this study are truly representative of
  the general U.S. population.
  \end{abstract}
  
 \end{frontmatter}

\hypertarget{introduction}{%
\section{Introduction}\label{introduction}}

\hypertarget{epidemiology-of-high-cholesterol}{%
\subsubsection{Epidemiology of High
Cholesterol}\label{epidemiology-of-high-cholesterol}}

The NIH's National Heart, Lung, and Blood Institute defines high
cholesterol, or hypercholesterolemia, as having total blood cholesterol
(TC) levels greater than or equal to 200 mg/dL
{[}\protect\hyperlink{ref-nhlbi2001}{1}{]}. More than one third of the
United States (U.S.) adult population has high TC and therefore are at a
heightened risk for cardiovascular disease (CVD), stroke, and death
{[}\protect\hyperlink{ref-virani2021}{2}{]}. Established risk factors of
high cholesterol include obesity, lack of physical activity, a diet high
in saturated and trans fats, type 2 diabetes, smoking, increasing age,
male gender, and familial hypercholesterolemia
{[}\protect\hyperlink{ref-cdc2020}{3}{]}. The prevalence of high
cholesterol in those who are obese has been recorded to be 2.6 times
greater than the prevalence of those with standard body mass index (BMI)
{[}\protect\hyperlink{ref-salazar2017}{4}{]}. Persons with type 2
diabetes have been observed to have 1.6-2.4 times higher prevalence of
dyslipidemia compared to persons without diabetes
{[}\protect\hyperlink{ref-jacobs2005}{5}{]}.

According to the American Heart Association's Heart Disease and Stroke
Statistics---2021 Update, adults aged 20 years or older in 2015-2018 had
a mean TC of 190.6 mg/dL {[}\protect\hyperlink{ref-virani2021}{2}{]}.
Healthy People, an evidence-based initiative that sets national 10-year
goals to improve the population health of the U.S., chose to target a
mean TC of 177.9 mg/dL by the year 2020
{[}\protect\hyperlink{ref-healthy2020}{6}{]}. Therefore, the mean TC
during 2015-2018 was not within the Healthy People goal for 2020. It was
also estimated that 93.9 million, or 38.1\%, of U.S. adults had high TC
{[}\protect\hyperlink{ref-virani2021}{2}{]}. The estimated number of
adults with very high TC was 28 million, or 11.5\%. Although the
prevalence of high TC in adults had decreased from 18.3\% in 2000 to
10.5\% in 2018, the report states that the decline is likely due to
greater uptake of medications rather than changes in lifestyle. The
estimated economic burden of managing and preventing high cholesterol in
the U.S. is between \$18.5 million and \$77 million every year. The
estimated burden of CVD, an outcome from having high cholesterol, is
\$219 billion {[}\protect\hyperlink{ref-ferrara2021}{7}{]}.

\hypertarget{working-hours}{%
\subsubsection{Working Hours}\label{working-hours}}

The U.S. Bureau of Labor Statistics reported full-time employees spent
an average of 8.78 hours per workday on work or work-related activities
{[}\protect\hyperlink{ref-bls2019}{8}{]}. This becomes 43.9 hours
throughout a 5-day workweek. Globally, 36.1\% of workers exceed 48 hours
each week {[}\protect\hyperlink{ref-rivera2020}{9}{]}. These numbers
surpass many developed countries' standards and recommendations. The
Australian government mandates employers must not work their employees
over 38 hours per week without reason
{[}\protect\hyperlink{ref-reynolds2018}{10}{]}. The European Union
states that employers ensure their workers do not exceed an average of
48 hours per week including overtime
{[}\protect\hyperlink{ref-euro2021}{11}{]}. The U.S. Fair Labor and
Standards Act {[}\protect\hyperlink{ref-flsa1938}{12}{]} requires
overtime pay after 40 hours in a week for nonexempt employees, and there
is no maximum number of hours.

When defining long working hours in research, prior studies have used
various cut points such as \textgreater40 per week
{[}\protect\hyperlink{ref-rivera2020}{9}{]}, \textgreater80 hours per
week {[}\protect\hyperlink{ref-lee2016}{13}{]}, and \textgreater11 hours
per day {[}\protect\hyperlink{ref-lemke2017}{14}{]}. According to the
U.S. Patient Protection and Affordable Care Act
{[}\protect\hyperlink{ref-ppaca2010}{15}{]}, employees working at least
30 hours per week on average are considered full-time. Working long
hours leaves less time for reaching the CDC's guidelines of recommended
exercise, diet, and sleep
{[}\protect\hyperlink{ref-artazcoz2009}{16}{]}. It also can expose
individuals to greater amounts of work strain and psychological stress.
These experiences have been shown to increase other biomarkers such as
blood pressure and heart rate instability
{[}\protect\hyperlink{ref-kivimaki2018}{17}{]}.

\hypertarget{literature-review}{%
\subsubsection{Literature review}\label{literature-review}}

Japanese engineers in the machinery manufacturing industry were observed
to have an age interaction when comparing TC across three levels of
weekly working hours {[}\protect\hyperlink{ref-sasaki1999}{18}{]}.
Participants that were aged 30-39 and worked 57-63 hours in a week had
17 mg/dL higher mean TC than participants of the same age group that
worked less than 57 hours (p\textless.05). This age and working hours
group also had 19 mg/dL higher mean TC than participants working more
than 63 hours each week (p\textless.05). Within participants aged 40-49
years, those working fewer than 57 hours had 41 mg/dL higher mean TC
than those working 57-63 hours (p\textless.001) and 34 mg/dL higher mean
TC than those working more than 63 hours (p\textless.01). All other age
groups had no statistical differences within their various levels of
working hours.

Researchers examining data from the Korean NHANES data showed 1.8 times
higher odds of coronary heart disease (CHD) for men (95\% CI: 1.2 --
2.5) and 1.6 times higher odds for women (95\% CI: 1.0 -- 2.6) working
more than 80 hours per week compared to those working 40 hours per week
{[}\protect\hyperlink{ref-lee2016}{13}{]}. The risk of stroke in these
two working groups for men was not statistically significant, but women
working more than 80 hours had 2.3 times higher odds of stroke (95\% CI:
1.5 -- 3.7) than women working 40 hours. Lee et al.~also noted that
results for men showed 1.8 times higher odds of CHD (95\% CI: 1.3 --
2.4) for those working less than 30 hours each week compared to those
working 40 hours.

A meta-analysis of 25 cohort studies across Europe, the U.S., and
Australia observed a 27\% increased risk of stroke (95\% CI: 1.03 --
1.56) in working 49-54 hours per week compared to 35-40 hours
{[}\protect\hyperlink{ref-kivimaki2015}{19}{]}. The risk of those
working more than 55 hours was 33\% greater than the 35-40 hours group
(95\% CI: 1.11 -- 1.61). The results from Kivimaki et al.~represented
528,908 men and women from cohort studies between 1975-2013 that were
initially free from stroke.

In the U.S., a cross-sectional study of truck drivers used linear
regression to estimate the effects of various factors such as age, daily
working hours, and sleep quality on TC, LDL-C, and HDL-C measurements
{[}\protect\hyperlink{ref-lemke2017}{14}{]}. Working greater than 11
hours daily was associated with an increase in LDL-C by 14.2 mg/dL (95\%
CI: 3.2 -- 25.3 mg/dL). No statistically significant association was
observed on TC or HDL-C.

A 22-year follow-up of 873 participants from the Western Australian
Pregnancy Cohort of 1981-1991 performed a cross-sectional analysis
comparing differences in cholesterol profiles
{[}\protect\hyperlink{ref-reynolds2018}{10}{]}. Reynolds et al.~resulted
with no statistically significant difference in mean triglyceride
measurements of those who were working more than 38 hours each week
(98.3 mg/dL) compared to those working 38 hours or less (95.7 mg/dL).
Reynolds et al.~observed that mean high-density lipoprotein cholesterol
(HDL-C) in the group working greater hours was lower by 4.3 mg/dL
(p\textless.001). These results were adjusted for education level, shift
work status, workload, and smoking.

Cross-sectional data on French participants of the CONSTANCES cohort
showed similar results after adjusting for age, socioeconomic status,
alcohol use, physical activity, depression symptoms, and chronic disease
{[}\protect\hyperlink{ref-virtanen2019}{20}{]} Men who were currently
working days longer than 10 hours had standardized means statistically
higher in TC (0.03, 95\% CI: 0.00 -- 0.06), higher in low-density
lipoprotein cholesterol (LDL-C) (0.04, 95\% CI: 0.01 -- 0.07), and lower
in HDL-C (-0.04, 95\% CI: -0.01 -- -0.07), compared to men who had never
worked longer than 10 hours in a day. Results for women and all
cholesterol measurements were not statistically significant. The
standardized mean differences in women currently exposed to those never
exposed were 0.00 for TC (95\% CI: -0.03 -- 0.03), -0.01 for LDL-C
(-0.05 -- 0.02), and 0.00 for HDL-C (-0.03 -- 0.04).

\hypertarget{gaps-in-the-literature}{%
\subsubsection{Gaps in the literature}\label{gaps-in-the-literature}}

Very few studies that examined associations between working hours and
cholesterol health were generalizable to the whole U.S. population. Some
methods utilized population cohorts in other comparable countries, but
cultural and sociologic differences cannot be ignored before making
assumptions. Studies that have used U.S. residents often regard
industry-specific populations or unusually high weekly working hours.
These limitations prevent much of the previous findings from reflecting
the U.S. population at large.

\hypertarget{summary}{%
\subsubsection{Summary}\label{summary}}

Many U.S. adults are working more than the standardized 40-hour
schedule, and thus reducing time for healthy behaviors such as sleep and
exercise. More time spent working also increases exposure to job-related
stress. Nearly 40\% of U.S. adults are affected by high cholesterol,
which comes as a result of unhealthy behaviors and stress. Prior
research has examined long working hours and its association with blood
cholesterol levels, but few are generalizable to most U.S. adults. This
analysis is unique in that it studies a random selection of working
adults to produce results that can represent the general U.S. adult
population.

\hypertarget{research-question}{%
\subsubsection{Research question}\label{research-question}}

Are greater self-reported hours worked in the prior week associated with
higher serum measurements of total cholesterol among employed men and
women aged 18-80 years who participated in NHANES 2017-2018?
\textbf{H0:} There is not an association between greater working hours
in the past week and high total blood cholesterol. \textbf{H1:} There is
a significant positive association between greater working hours in the
past week and high total blood cholesterol.

\hypertarget{methods}{%
\section{Methods}\label{methods}}

\hypertarget{study-design-and-setting}{%
\subsubsection{Study design and
setting}\label{study-design-and-setting}}

The National Center for Health Statistics (NCHS) administered the
National Health and Nutrition Examination Survey (NHANES) in-person to
adults and adolescents in the U.S. during 2017 and 2018. This survey
recorded cross-sectional health information and measurements for the
purpose of estimating nationwide disease prevalence and aiding in health
policy development. The sample design was a complex multi-level process
that oversampled and undersampled certain demographics that later became
weighted to represent the whole noninstitutionalized civilian population
of the United States. Participants were interviewed about their personal
demographics, health status, and behaviors, while measurements were
recorded by a mobile clinic during a standardized physical examination.

\hypertarget{study-population}{%
\subsubsection{Study population}\label{study-population}}

During 2017-2018 the National Health and Nutrition Examination Survey
(NHANES) recruited 9,254 males and females of all ages, including
children with parental approval. Of these participants, 1,715 were
selected based on their blood tests and questionnaire results. All
selected observations had completed TC measurements, worked at least 30
total hours at all jobs during the week prior to being surveyed, were
not currently taking cholesterol medication, and were at least 18 years
of age. Any participants who refused to answer or were unable to
remember their prior week's total work hours were not included. Also
participants missing data for their BMI and income-to-poverty ratio were
removed as this information was needed in the multivariable analysis.

\hypertarget{data-sources-and-measurement}{%
\subsubsection{Data sources and
measurement}\label{data-sources-and-measurement}}

Outcome: TC was recorded by a combined effort of collecting blood
samples by the mobile examination clinic and enzymatic assay methods
performed by contracted laboratories. Collection of the samples occurred
immediately prior to the questionnaire. After the completion of the
laboratory analyses, TC was recorded as continuous values with units of
milligrams per deciliter (mg/dL) of blood. No exclusions were made
solely based on TC values.

Exposure: Total working hours in the week prior to the survey
administration was obtained from a series of questions. First
participants were asked ``In this part of the survey I will ask you
questions about your work experience. Which of the following were you
doing last week?'' The options to answer this question were as followed:
Working at a job or business; with a job or business but not at work;
looking for work; not working at a job or business; refused; or don't
know. Only those who responded that they worked at a job or business
were considered for this study. A follow-up question was asked to those
who reported that they worked: ``How many hours did you work last week
at all jobs or businesses?'' The NHANES administrators recorded those
who answered between 1 and 5 hours as just ``5.'' Six to 78 hours were
recorded as discrete values. No respondent reported 79 hours, and those
who reported 80 or more were recorded as ``80'' per the NHANES research
methods. The goal of this study was to examine full-time workers,
therefore observations were excluded from the analysis if they worked
less than 30 hours, refused to report, or did not know how many hours
they worked last week.

Covariates: Age was recorded at the time of screening and was coded as
discrete values of 0-79, and 80 being a topcode representing individuals
80 years of age and older. BMI was a calculated continuous variable
(kg/m2) using weight and height measurements taken by the NHANES
examiners. Gender was categorized as either male or female, and there
were not any missing values in the dataset. Participants were asked if a
doctor has ever told them they have high cholesterol in the following
format: ``Have you ever been told by a doctor or other health
professional that your blood cholesterol level was high?'' Responses
were recorded as either ``Yes,'' ``No,'' ``Refused,'' or ``Don't Know.''
For this study, ``Refused'' and ``Don't know'' were collapsed into
missing values. Participants were asked to recall how long they spend
sitting down each day with the following question: ``How much time do
you usually spend sitting on a typical day?'' Responses were recorded as
either discrete values, ``Refused,'' or ``Don't Know.'' The ``Refused''
and ``Don't Know'' responses were collapsed into missing values.
Vigorous physical activity at work was measured as ``Yes,'' ``No,'' or
``Don't Know'' with the following question: ``Does your work involve
vigorous-intensity activity that causes large increases in breathing or
heart rate like carrying or lifting heavy loads, digging or construction
work for at least 10 minutes continuously?'' The response ``Don't Know''
was recoded as missing for this study. Vigorous recreational activity
was also inquired in the following format: ``In a typical week do you do
any vigorous-intensity sports, fitness, or recreational activities that
cause large increases in breathing or heart rate like running or
basketball for at least 10 minutes continuously?'' Responses were coded
similarly to the vigorous work activity variable. Participants
self-selected their race or ethnicity from these options: Mexican
American; Other Hispanic; Non-Hispanic White; Non-Hispanic Black;
Non-Hispanic Asian; and Other Race Including Multi-Racial. Mexican
American and Other Hispanic were collapsed into a single category called
``Hispanic'' for this study. Education level was acquired with the
following question: ``What is the highest grade or level of school you
have completed or the highest degree you have received?'' Available
responses were: less than 9th grade; 9-11th grade; high school
graduate/GED or equivalent; some college or AA degree; college graduate
or above; refused; don't know; or missing. For this study, ``less than
9th grade'' and ``9-11th grade'' were collapsed into a new category
called ``No high school diploma or equivalent.'' The ``Refused'' and
``Don't Know'' responses were recategorized into missing values. Lastly,
Income-to-Poverty Ratio was a calculated continuous variable using the
ratio of self-reported household income levels to the participant's
local poverty line. The value ``5'' was a topcode for any value greater
than or equal to 5. The purpose of this topcode was to retain anonymity
of participants with very high incomes.

\hypertarget{efforts-to-address-bias}{%
\subsubsection{Efforts to address bias}\label{efforts-to-address-bias}}

Those who worked less than full-time in the prior week, defined as less
than 30 hours, were removed from the study to prevent a potential bias
that could come from observations who had not worked at all or worked
very little. Children and adolescents were removed as they may not have
lived long enough to observe direct effects of working hours on TC. The
final model controlled for the effects of the selected covariates to
estimate the unbiased association between prior week working hours and
TC.

\hypertarget{statistical-methods}{%
\subsubsection{Statistical methods}\label{statistical-methods}}

Descriptive and analytic statistics were calculated with R 4.1.1 ``Kick
Things'' and weights included with the NHANES 2017-2018 dataset were
utilized. Unweighted analyses were not performed. Alpha was set to 0.05
for all interpretations. Means and standard errors were reported for
normally distributed variables while medians and ranges were reported
for non-normal distributions.

Simple linear regression with survey weights was used to determine the
effects of the continuous variables on TC for the bivariate analysis.
Pearson's correlations were used to describe the strength of these
associations. Bartlett's test for homogeneity was used to determine the
inclusion of a categorical variable in the bivariate analysis.
Independent t-tests and ANOVA were performed to calculate differences in
means for categorical variables. Multiple linear regression with survey
weights was used in the adjusted model to determine the effect of prior
week working hours on TC. Forward stepwise selection determined
inclusion of covariates for the adjusted model. A 10\% minimum change in
effect of working hours was used as a cutoff for this selection method.

A post-hoc analysis was performed to measure a potential bias resulting
from a healthy worker effect. Pearson's X2 tests were used to compare
high TC (\textgreater200 mg/dL) prevalence and very high TC
(\textgreater240 mg/dL) prevalence between those included and those
excluded in the current study population. An independent t-test
determined the difference in mean TC between the two groups.

\hypertarget{results}{%
\section{Results}\label{results}}

\hypertarget{key-findings}{%
\subsubsection{Key findings}\label{key-findings}}

\begin{itemize}
\tightlist
\item
  37.0\% of the weighted population had clinically high total blood
  cholesterol.
\item
  47.0\% reported working more than 40 hours in the prior week.
\item
  Each hour worked was associated with a decrease of total blood
  cholesterol by 0.28 mg/dL after adjusting for covariates.
\end{itemize}

\hypertarget{study-population-1}{%
\subsubsection{Study population}\label{study-population-1}}

The NHANES 2017-2018 dataset consisted of 9,254 observations that was
reduced to 1,715 observations due to the exclusion criteria of this
study. The final study population was weighted to represent 95,960,477
non-institutionalized adults in the United States. The weighted
interquartile range for age was 29 -- 51 years old. A BMI of 25 or
greater was recorded for 72.9\% of the weighted population (SE = 1.6\%).
The prevalence of either college attendance or completion of an advanced
degree was 66.2\% (SE = 1.7\%). Household income at or below poverty
lines was calculated in 9.0\% of the weighted population (SE = 0.7\%).

\hypertarget{total-blood-cholesterol}{%
\subsubsection{Total blood cholesterol}\label{total-blood-cholesterol}}

TC was normally distributed amongst the weighted population and had a
mean of 190.0 mg/dL (SE = 1.4). Approximately 37.0\% had TC \(/geqq\)
200 mg/dL (SE = 1.8\%) and 9.9\% had TC \(\geqq\) 240 mg/dL or higher
(SE = 1.1\%). When considering the Healthy People 2020 target mean TC,
38.5\% of the population were at or under 177.9 mg/dL (SE = 1.7\%).

\hypertarget{working-hours-1}{%
\subsubsection{Working hours}\label{working-hours-1}}

Approximately 33.6\% of the weighted population worked exactly 40 hours
(SE = 1.7\%) and 46.9\% worked more than 40 hours (SE = 1.8\%). The
proportion of those who worked between 30-39 hours was 19.4\% (SE =
1.4\%). Male mean working hours was 46.7 hours (SE = 0.6) whereas the
female mean was 42.8 hours (SE = 0.5).

\hypertarget{bivariate-analysis}{%
\subsubsection{Bivariate analysis}\label{bivariate-analysis}}

Prior week working hours had a weak negative unadjusted correlation with
TC (R = -0.08) and accounted for a decrease in TC by 0.31 mg/dL for
every hour worked (95\% CI = -0.53 -- -0.09). The strongest positive
unadjusted correlation with TC were age (R = 0.31) and BMI (R = 0.12).
An increase of 1 year in age was associated with an increase of 0.93
mg/dL (95\% CI = 0.72 -- 1.1) in TC. For BMI, an increase of 1 unit was
associated with an increase of TC by 0.65 mg/dL (95\% CI = 0.32 --
0.97). Taking part in vigorous recreational physical activity was
associated with a reduction in mean TC by 11.1 mg/dL (95\% CI = 5.8 --
16.3). Non-Hispanic Blacks had the lowest mean TC (182.5 mg/dL, SE =
2.2) while the Other and Multi-racial category had the highest mean TC
(200.0 mg/dL, SE = 5.3). Higher levels of education generally decreased
mean TC, but these differences were not statistically significant.

\hypertarget{multivariable-analysis}{%
\subsubsection{Multivariable analysis}\label{multivariable-analysis}}

The adjusted model more accurately predicted the variance in TC
(Adjusted R2 = 0.12) compared to the unadjusted model of prior week
working hours (Adjusted R2 = 0.006), though both were not strong
predictors. Each hour worked was associated with an average reduction in
TC by 0.28 mg/dl (95\% CI = -0.44 -- -0.11) after adjusting for age,
BMI, vigorous recreational activity, income-to-poverty ratio, race, and
ethnicity. Age and BMI had the largest effects on TC in the adjusted
model. An increase in 1-year of age was associated with an increase in
mean TC by 0.8 mg/dL (0.7 -- 1.0), and a 1-unit increase of BMI was
associated with an increase in mean TC by 0.5 mg/dL (0.2 -- 0.7).

\hypertarget{post-hoc-analysis}{%
\subsubsection{Post-hoc Analysis}\label{post-hoc-analysis}}

Clinically high TC was prevalent in 37.0\% of the study population and
41.6\% of the excluded population. Clinically high TC was not
independent from the two populations (X2 = 7.8, p=.06). Very high TC was
prevalent in 9.9\% of the study population and 12.3\% of the excluded
population. This variable was also not independent from the two
populations (X2 = 2.2, p=.14). Mean TC in the study population was 2.9
mg/dL lower than that of the excluded population (95\% CI: -6.6 --
0.84).

\hypertarget{discussion}{%
\section{Discussion}\label{discussion}}

\hypertarget{principal-findings}{%
\subsubsection{Principal Findings}\label{principal-findings}}

More than 1 in 3 represented U.S. working adults in this study had
clinically high cholesterol, and nearly half of the study population
worked more than 40 hours in the prior week. The hypothesis that a
positive association between working hours and TC was not supported, and
rather a negative association was observed. Although significant, the
predictability of working hours on TC was weak.

\hypertarget{comparison-to-other-studies}{%
\subsubsection{Comparison to Other
Studies}\label{comparison-to-other-studies}}

Past research has had varying results such as no associations or
significant declining of cholesterol health with higher working hours.
Sisaki, Iwasaki, and Hisanaga
{[}\protect\hyperlink{ref-sasaki1999}{18}{]} observed a mean TC of 202
mg/dL and a mean of 60 hours worked in a week for their study
population. Although their mean TC is similar to this study, the
difference in mean working hours is notable. Their research found that
observations aged 30-39 working in the mid-length group (57-63 hours
each week) had higher TC than the same age group working over 63 hours.
This could possibly be another healthy worker effect resulting with
healthier people being capable of working longer hours.

The Reynolds et al {[}\protect\hyperlink{ref-reynolds2018}{10}{]} cohort
study resulted with no difference in TC between those who worked 38
hours or fewer in a week compared to those who worked more than 38 hours
in a week. They did note that HDL-C was significantly lower in the group
that worked more hours. Their determination of working hours may have
been more accurate than the NHANES method because in Reynolds et al they
asked participants about their usual workweek length as opposed to the
most recent work week. This difference changes the hypothesized
mechanism from an immediate effect to a long-term exposure effect.

On the other hand, Virtanen et al.
{[}\protect\hyperlink{ref-virtanen2019}{20}{]} resulted with males
currently working more than 10 hours in a day having higher TC than
males who had never worked more than 10 hours in a day. This difference
was mostly attributed to the men who were currently working more than 10
hours in a day that also had over 15 years of this exposure. Fewer years
of exposure reduced the strength of this effect. There was no
significance in the difference with their female participants. Virtanen
et al.~had observed a slightly older population from France, with the
mean age being 48 years old compared to the median age of 39 in this
NHANES study. Comparability is also affected by the difference in how
working hours data was collected. Virtanen et al.~used daily working
hours rather than weekly hours. Weekly hours cannot be calculated
because the number of days worked each week are not reported.

Lee et al. {[}\protect\hyperlink{ref-lee2016}{13}{]} may have also had a
healthy worker effect. Their study resulted with risk of CHD increasing
1.8 times for those working less than 30 hours each week when compared
to those working 40 hours each week. This could potentially be caused by
healthy participants being physically able to work full-time, while
unhealthy participants were underemployed due to their health status.

\hypertarget{strengths-limitations-and-bias}{%
\subsubsection{Strengths, Limitations, and
Bias}\label{strengths-limitations-and-bias}}

Working hours and total cholesterol as continuous measurements preserved
analytical power when computing associations. This allowed for the use
of a linear regression model that would not be affected by biases in
categorization. Weighting observations increased the generalizability to
a large portion of the U.S. working adult population. Inclusion of
physical activity inside and outside workplace environments increased
accuracy of the adjusted association between working hours and total
cholesterol. Also, using income-to-poverty ratios rather than raw
household income accounts for differences in cost of living due to
geographic location. Limitations from NHANES 2017-2018 were inherited to
this study as this was the sole source of data. Much of the data was
collected as self-reported information and accuracy was subject to the
participants' responses. Blood specimen collection only occurred once
and working hours information was asked of only a single week, which can
leave randomness unidentifiable. The cross-sectional design of this
study prevents the determination of causal associations. Since only one
measurement of TC and working hours was recorded for each participant,
the result of that one recording may not represent the truth. The
implication of this is that participants could be misclassified based on
that one result and put them into categories that they otherwise would
not belong in. Also, by limiting the study population to only those
participants who are currently working at the time of the NHANES
2017-2018 data collection, the results of this study may be influenced
by the healthy worker effect. Those observations that worked many hours
may have only been able to do so because of good health, meaning it is
possible that unobserved persons who were negatively affected had to
reduce their hours or even stop working entirely.

\hypertarget{findings-implications}{%
\subsubsection{Findings Implications}\label{findings-implications}}

The results of this study and comparisons to past research suggest that
TC may not be at risk of worsening due to long working hours within the
general U.S. population. When researching health effects of work week
standards, it may be more important to examine other biomarkers such as
blood pressure or micronutrient deficiency. These implications are
generalizable to the working adult-aged population in the U.S. because
of the weighting method used in the NHANES dataset. The post-hoc
analysis observed that although the excluded population had 4.6\%
greater prevalence of clinically high TC compared to the study
population, this difference was not statistically significant (p=.06).
The results though are close enough to warrant further study. If future
results have a similar or greater difference, then it could be confirmed
that a healthy worker effect occurs with this study design. This effect
can be controlled for by included participants who are no longer
working, or working reduced hours, and inquiring about their past work
exposures. The cross-sectional design and nature of the questionnaire
leaves the long-term exposure associations unaddressed. Therefore,
future study into working hours and TC should adopt a longitudinal
design in which multiple blood samples are collected and working hours
are averaged over time. Additionally, the healthy worker effect could be
controlled for by including persons who are no longer working or
underemployed. Other effects of working long hours could be identified
by asking participants questions regarding specific experiences at work
such as stress or eating habits.

\hypertarget{conclusion}{%
\section{Conclusion}\label{conclusion}}

Increased working hours were associated with decreased TC. This result
differs from past research that concluded with either no association or
a positive association. A potential explanation could be that the
healthy worker effect caused a bias in this direction due to healthier
observations working longer hours. If future results confirm the
existence of this bias, then researchers need to consider under- or
unemployed populations whose health status affects their ability to work
full-time.

\hypertarget{references}{%
\section*{References}\label{references}}
\addcontentsline{toc}{section}{References}

\hypertarget{refs}{}
\begin{CSLReferences}{0}{0}
\leavevmode\vadjust pre{\hypertarget{ref-nhlbi2001}{}}%
\CSLLeftMargin{{[}1{]} }
\CSLRightInline{NHLBI. {ATP III At-A-Glance}: {Quick Desk Reference}.
National Institutes of Health 2001.}

\leavevmode\vadjust pre{\hypertarget{ref-virani2021}{}}%
\CSLLeftMargin{{[}2{]} }
\CSLRightInline{Virani SS, Alonso A, Aparicio HJ, Benjamin EJ,
Bittencourt MS, Callaway CW, et al. Heart {Disease} and {Stroke
Statistics}\textemdash 2021 {Update}. Circulation 2021;143:e254--743.
doi:\href{https://doi.org/10.1161/CIR.0000000000000950}{10.1161/CIR.0000000000000950}.}

\leavevmode\vadjust pre{\hypertarget{ref-cdc2020}{}}%
\CSLLeftMargin{{[}3{]} }
\CSLRightInline{CDC. Knowing {Your Risk}: {High Cholesterol}. Centers
for Disease Control and Prevention 2020.}

\leavevmode\vadjust pre{\hypertarget{ref-salazar2017}{}}%
\CSLLeftMargin{{[}4{]} }
\CSLRightInline{Salazar MR, Carbajal HA, Espeche WG, Aizpurúa M,
Marillet AG, Leiva Sisnieguez CE, et al. Use of the
triglyceride/high-density lipoprotein cholesterol ratio to identify
cardiometabolic risk: Impact of obesity? Journal of Investigative
Medicine 2017;65:323.
doi:\url{http://dx.doi.org/10.1136/jim-2016-000248}.}

\leavevmode\vadjust pre{\hypertarget{ref-jacobs2005}{}}%
\CSLLeftMargin{{[}5{]} }
\CSLRightInline{Jacobs MJ, Kleisli T, Pio JR, Malik S, L'Italien GJ,
Chen RS, et al. Prevalence and control of dyslipidemia among persons
with diabetes in the {United States}. Diabetes Research and Clinical
Practice 2005;70:263--9.
doi:\href{https://doi.org/10.1016/j.diabres.2005.03.032}{10.1016/j.diabres.2005.03.032}.}

\leavevmode\vadjust pre{\hypertarget{ref-healthy2020}{}}%
\CSLLeftMargin{{[}6{]} }
\CSLRightInline{Heart {Disease} and {Stroke} \textbar{} {Healthy People}
2020 n.d.}

\leavevmode\vadjust pre{\hypertarget{ref-ferrara2021}{}}%
\CSLLeftMargin{{[}7{]} }
\CSLRightInline{Ferrara P, Laura DD, Cortesi PA, Mantovani LG. The
economic impact of hypercholesterolemia and mixed dyslipidemia: {A}
systematic review of cost of illness studies. PLOS ONE 2021;16:e0254631.
doi:\href{https://doi.org/10.1371/journal.pone.0254631}{10.1371/journal.pone.0254631}.}

\leavevmode\vadjust pre{\hypertarget{ref-bls2019}{}}%
\CSLLeftMargin{{[}8{]} }
\CSLRightInline{Table 4. {Employed} persons working and time spent
working on days worked by full- and part-time status and sex, jobholding
status, educational attainment, and day of week, annual averages n.d.}

\leavevmode\vadjust pre{\hypertarget{ref-rivera2020}{}}%
\CSLLeftMargin{{[}9{]} }
\CSLRightInline{Rivera AS, Akanbi M, O'Dwyer LC, McHugh M. Shift work
and long work hours and their association with chronic health
conditions: {A} systematic review of systematic reviews with
meta-analyses. PLOS ONE 2020;15:e0231037.
doi:\href{https://doi.org/10.1371/journal.pone.0231037}{10.1371/journal.pone.0231037}.}

\leavevmode\vadjust pre{\hypertarget{ref-reynolds2018}{}}%
\CSLLeftMargin{{[}10{]} }
\CSLRightInline{Reynolds AC, Bucks RS, Paterson JL, Ferguson SA, Mori
TA, McArdle N, et al. Working (longer than) 9 to 5: Are there
cardiometabolic health risks for young {Australian} workers who report
longer than 38-h working weeks? International Archives of Occupational
and Environmental Health 2018;91:403--12.
doi:\href{https://doi.org/10.1007/s00420-018-1289-4}{10.1007/s00420-018-1289-4}.}

\leavevmode\vadjust pre{\hypertarget{ref-euro2021}{}}%
\CSLLeftMargin{{[}11{]} }
\CSLRightInline{Working hours in {EU}: {What} are the minimum standards?
Your Europe 21AD.}

\leavevmode\vadjust pre{\hypertarget{ref-flsa1938}{}}%
\CSLLeftMargin{{[}12{]} }
\CSLRightInline{Fair {Labor Standards Act} 1938:1060--70.}

\leavevmode\vadjust pre{\hypertarget{ref-lee2016}{}}%
\CSLLeftMargin{{[}13{]} }
\CSLRightInline{Lee D-W, Hong Y-C, Min K-B, Kim T-S, Kim M-S, Kang M-Y.
The effect of long working hours on 10-year risk of coronary heart
disease and stroke in the {Korean} population: The {Korea National
Health} and {Nutrition Examination Survey} ({KNHANES}), 2007 to 2013.
Annals of Occupational and Environmental Medicine 2016;28.
doi:\href{https://doi.org/10.1186/s40557-016-0149-5}{10.1186/s40557-016-0149-5}.}

\leavevmode\vadjust pre{\hypertarget{ref-lemke2017}{}}%
\CSLLeftMargin{{[}14{]} }
\CSLRightInline{Lemke MK, Apostolopoulos Y, Hege A, Wideman L, Sönmez S.
Work, sleep, and cholesterol levels of {U}.{S}. Long-haul truck drivers.
Industrial Health 2017;55:149--61.
doi:\href{https://doi.org/10.2486/indhealth.2016-0127}{10.2486/indhealth.2016-0127}.}

\leavevmode\vadjust pre{\hypertarget{ref-ppaca2010}{}}%
\CSLLeftMargin{{[}15{]} }
\CSLRightInline{The {Patient Protection} and {Affordable Care Act}
2010:119--025.}

\leavevmode\vadjust pre{\hypertarget{ref-artazcoz2009}{}}%
\CSLLeftMargin{{[}16{]} }
\CSLRightInline{Artazcoz L, Cortès I, Escribà-Agüir V, Cascant L,
Villegas R. Understanding the relationship of long working hours with
health status and health-related behaviours. Journal of Epidemiology and
Community Health (1979-) 2009;63:521--7.}

\leavevmode\vadjust pre{\hypertarget{ref-kivimaki2018}{}}%
\CSLLeftMargin{{[}17{]} }
\CSLRightInline{Kivimäki M, Steptoe A. Effects of stress on the
development and progression of cardiovascular disease. Nature Reviews
Cardiology 2018;15:215--30.
doi:\href{https://doi.org/10.1038/nrcardio.2017.189}{10.1038/nrcardio.2017.189}.}

\leavevmode\vadjust pre{\hypertarget{ref-sasaki1999}{}}%
\CSLLeftMargin{{[}18{]} }
\CSLRightInline{Sasaki T, Iwasaki K, Oka T, Hisanaga N. Association of
{Working Hours} with {Biological Indices Related} to the {Cardiovascular
System} among {Engineers} in a {Machinery Manufacturing Company}.
INDUSTRIAL HEALTH 1999;37:457--63.
doi:\href{https://doi.org/10.2486/indhealth.37.457}{10.2486/indhealth.37.457}.}

\leavevmode\vadjust pre{\hypertarget{ref-kivimaki2015}{}}%
\CSLLeftMargin{{[}19{]} }
\CSLRightInline{Kivimäki M, Jokela M, Nyberg ST, Singh-Manoux A,
Fransson EI, Alfredsson L, et al. Long working hours and risk of
coronary heart disease and stroke: A systematic review and meta-analysis
of published and unpublished data for 603 838 individuals. The Lancet
2015;386:1739--46.
doi:\href{https://doi.org/10.1016/S0140-6736(15)60295-1}{10.1016/S0140-6736(15)60295-1}.}

\leavevmode\vadjust pre{\hypertarget{ref-virtanen2019}{}}%
\CSLLeftMargin{{[}20{]} }
\CSLRightInline{Virtanen M, Magnusson Hansson L, Goldberg M, Zins M,
Stenholm S, Vahtera J, et al. Long working hours, anthropometry, lung
function, blood pressure and blood-based biomarkers: Cross-sectional
findings from the {CONSTANCES} study. Journal of Epidemiology and
Community Health 2019;73:130--5.
doi:\href{https://doi.org/10.1136/jech-2018-210943}{10.1136/jech-2018-210943}.}

\end{CSLReferences}


\end{document}
