\documentclass[]{elsarticle} %review=doublespace preprint=single 5p=2 column
%%% Begin My package additions %%%%%%%%%%%%%%%%%%%
\usepackage[hyphens]{url}

  \journal{N/A} % Sets Journal name


\usepackage{lineno} % add
\providecommand{\tightlist}{%
  \setlength{\itemsep}{0pt}\setlength{\parskip}{0pt}}

\usepackage{graphicx}
%%%%%%%%%%%%%%%% end my additions to header

\usepackage[T1]{fontenc}
\usepackage{lmodern}
\usepackage{amssymb,amsmath}
\usepackage{ifxetex,ifluatex}
\usepackage{fixltx2e} % provides \textsubscript
% use upquote if available, for straight quotes in verbatim environments
\IfFileExists{upquote.sty}{\usepackage{upquote}}{}
\ifnum 0\ifxetex 1\fi\ifluatex 1\fi=0 % if pdftex
  \usepackage[utf8]{inputenc}
\else % if luatex or xelatex
  \usepackage{fontspec}
  \ifxetex
    \usepackage{xltxtra,xunicode}
  \fi
  \defaultfontfeatures{Mapping=tex-text,Scale=MatchLowercase}
  \newcommand{\euro}{€}
\fi
% use microtype if available
\IfFileExists{microtype.sty}{\usepackage{microtype}}{}
\bibliographystyle{elsarticle-harv}
\ifxetex
  \usepackage[setpagesize=false, % page size defined by xetex
              unicode=false, % unicode breaks when used with xetex
              xetex]{hyperref}
\else
  \usepackage[unicode=true]{hyperref}
\fi
\hypersetup{breaklinks=true,
            bookmarks=true,
            pdfauthor={},
            pdftitle={WORKING DRAFT\ldots{} Is greater number of hours worked in the prior week associated with high total blood cholesterol among men and women ages 18-80 from NHANES 2017-2018 participants of the United States?},
            colorlinks=true,
            urlcolor=blue,
            linkcolor=magenta,
            pdfborder={0 0 0}}
\urlstyle{same}  % don't use monospace font for urls

\setcounter{secnumdepth}{0}
% Pandoc toggle for numbering sections (defaults to be off)
\setcounter{secnumdepth}{0}

% Pandoc citation processing
\newlength{\cslhangindent}
\setlength{\cslhangindent}{1.5em}
\newlength{\csllabelwidth}
\setlength{\csllabelwidth}{3em}
% for Pandoc 2.8 to 2.10.1
\newenvironment{cslreferences}%
  {}%
  {\par}
% For Pandoc 2.11+
\newenvironment{CSLReferences}[2] % #1 hanging-ident, #2 entry spacing
 {% don't indent paragraphs
  \setlength{\parindent}{0pt}
  % turn on hanging indent if param 1 is 1
  \ifodd #1 \everypar{\setlength{\hangindent}{\cslhangindent}}\ignorespaces\fi
  % set entry spacing
  \ifnum #2 > 0
  \setlength{\parskip}{#2\baselineskip}
  \fi
 }%
 {}
\usepackage{calc}
\newcommand{\CSLBlock}[1]{#1\hfill\break}
\newcommand{\CSLLeftMargin}[1]{\parbox[t]{\csllabelwidth}{#1}}
\newcommand{\CSLRightInline}[1]{\parbox[t]{\linewidth - \csllabelwidth}{#1}\break}
\newcommand{\CSLIndent}[1]{\hspace{\cslhangindent}#1}

% Pandoc header
\usepackage{setspace}\doublespacing



\begin{document}
\begin{frontmatter}

  \title{WORKING DRAFT\ldots{} Is greater number of hours worked in the
prior week associated with high total blood cholesterol among men and
women ages 18-80 from NHANES 2017-2018 participants of the United
States?}
    \author[San Diego State University School of Public
Health]{Christian Hicks}
   \ead{chicks5799@sdsu.edu} 
      
  \begin{abstract}
  This is the abstract.

  It consists of two paragraphs.
  \end{abstract}
  
 \end{frontmatter}

\hypertarget{introduction}{%
\section{Introduction}\label{introduction}}

\hypertarget{epidemiology-of-the-outcome}{%
\subsubsection{Epidemiology of the
outcome}\label{epidemiology-of-the-outcome}}

The NIH's National Heart, Lung, and Blood Institute defines high
cholesterol as having total blood cholesterol (TC) levels greater than
or equal to 200 mg/dL {[}\protect\hyperlink{ref-nhlbi2001}{1}{]}. More
than one third of the United States adult population has high
cholesterol and therefore are at a heightened risk for cardiovascular
disease (CVD) and stroke {[}\protect\hyperlink{ref-virani2021}{2}{]}.
Established risk factors of high cholesterol include obesity, lack of
physical activity, diet high in saturated and trans fat, type 2
diabetes, smoking, age, gender, and family history of high cholesterol
{[}\protect\hyperlink{ref-cdc2020}{3}{]}.

According to the American Heart Association's Heart Disease and Stroke
Statistics---2021 Update, adults aged 20 and older in 2015-2018 had a
mean TC of 190.6 mg/dL {[}\protect\hyperlink{ref-virani2021}{2}{]}. This
was not within the Healthy People 2020 target goal of 177.9 mg/dL. It
was also estimated that 93.9 million, or 38.1\%, of U.S. adults had
greater than 200 mg/dL TC. For TC greater than 240 mg/dL the estimated
affected adults were 28 million, or 11.5\%. Although the prevalence of
high TC in adults had decreased from 18.3\% in 2000 to 10.5\% in 2018,
the report states that the decline is likely due to greater uptake of
medications rather than changes in lifestyle. The estimated economic
burden of managing and preventing high cholesterol in the U.S. is
between \$18.5 million and \$77 million every year. The estimated burden
of CVD, an outcome of having high cholesterol, is \$219 billion
{[}\protect\hyperlink{ref-ferrara2021}{4}{]}.

\hypertarget{epidemiology-of-the-exposure}{%
\subsubsection{Epidemiology of the
exposure}\label{epidemiology-of-the-exposure}}

During 2019 in the U.S., the Bureau of Labor Statistics reported
full-time employees spent an average of 8.78 hours per workday on work
or work-related activities. This becomes 43.9 hours throughout a 5-day
workweek. Globally, 36.1\% of workers exceed 48 hours each week
{[}\protect\hyperlink{ref-rivera2020}{5}{]}. These numbers surpass many
developed countries' standards and recommendations. The Australian
government mandates employers must not work their employees over 38
hours per week without reason. The European Union states that employers
ensure their workers do not exceed an average of 48 hours per week
including overtime. The U.S. Fair Labor and Standards Act
{[}\protect\hyperlink{ref-flsa1938}{6}{]} requires overtime pay after 40
hours in a week for nonexempt employees, and there is no maximum number
of hours.

When defining long working hours in research, prior studies have used
various cut points such as \textgreater40 per week
{[}\protect\hyperlink{ref-rivera2020}{5}{]}, \textgreater80 hours per
week {[}\protect\hyperlink{ref-lee2016}{7}{]}, and \textgreater11 hours
per day {[}\protect\hyperlink{ref-lemke2017}{8}{]}. According to the
U.S. Patient Protection and Affordable Care Act
{[}\protect\hyperlink{ref-ppaca2010}{9}{]}, employees working at least
30 hours per week on average are considered full-time. Working long
hours leaves less time for reaching the CDC's guidelines of recommended
exercise, diet, and sleep
{[}\protect\hyperlink{ref-artazcoz2009}{10}{]}. It also can expose
individuals to greater amounts of work strain and psychological stress.
These experiences have been shown to increase other biomarkers such as
blood pressure and heart rate instability
{[}\protect\hyperlink{ref-kivimaki2018}{11}{]}.

\hypertarget{literature-review}{%
\subsubsection{Literature review}\label{literature-review}}

In 1999 a study examined blood cholesterol of engineers in Japan working
in machinery manufacturing {[}\protect\hyperlink{ref-sasaki1999}{12}{]}.
No significant differences were directly shown between the weekly
working hours of \textless57, 57-63, and \textgreater63. They did note
though that a stratification by age resulted with an interaction for the
age group of 30-39, but not the age group of 20-29. The group with 57-63
work hours had a higher mean TC than the other two groups.

A survey of random working adults in France described that exposure to
long working hours, defined as +10-hour days in at least 50 days out of
the year, was associated with adverse lipid levels in men
{[}\protect\hyperlink{ref-virtanen2019}{13}{]}. Their results were not
significant for women.

In a meta-analysis of long working hours, coronary heart disease, and
stroke resulted with a 27\% increased risk of stroke in working 49-54
hours per week compared to 36-40 hours
{[}\protect\hyperlink{ref-kivimaki2015}{14}{]}. The risk of those
working more than 55 hours was 33\% greater.

Researchers examining data from the Korean NHANES data showed 1.76 times
higher odds of coronary heart disease for men and 1.63 times higher odds
for women working \textgreater80 hours per week compared to those
working 40 hours per week {[}\protect\hyperlink{ref-lee2016}{7}{]}. The
risk of stroke in these two working groups for men was insignificant,
but women working \textgreater80 hours had 2.32 times higher odds of
stroke than women working 40 hours.

A cross-sectional study of U.S. truck drivers was performed to explore
an effect of long working hours and blood cholesterol measurements
{[}\protect\hyperlink{ref-lemke2017}{8}{]}. The study population had a
mean low-density lipoprotein cholesterol (LDL-C) of 113.66 mg/dL. When
designing a linear regression model, working greater than 11 hours per
day significantly increased the estimated LDL-C 14.24 mg/dL.

22-year-old Australians from the Western Australian Pregnancy Cohort
Study examined the effect of long working hours on blood cholesterol
measurements {[}\protect\hyperlink{ref-reynolds2018}{15}{]}. There were
no significant results of TC or LDL-C, but high-density lipoprotein
cholesterol (HDL-C) was lower in those working \textgreater38 hours
compared to those working 38 hours or less.

\hypertarget{gaps-in-the-literature}{%
\subsubsection{Gaps in the literature}\label{gaps-in-the-literature}}

Most prior studies of long working hours used CHD or stroke as the
outcome rather than blood cholesterol measurements. Of the studies that
did examine blood cholesterol they were either not done in the United
States, had very high cut points for defining long hours, or did not
describe comparisons in risk or odds.

\hypertarget{summary}{%
\subsubsection{Summary}\label{summary}}

U.S. adults spend an average of 8.78 hours per workday on work-related
activities, exceeding the standardized 40-hour workweek and reducing
available time for healthy behaviors such as sleep and exercise, and
increasing exposure to job-related stress. During 2013-2016 U.S. adults
had a mean TC of 190.6 mg/dL, which is above the Healthy People 2020
goal of 177.9 mg/dL. 93.9 million adults (38.1\%) had a TC of 200 mg/dL
or greater. 28 million adults (11.5\%) had a TC of 240 mg/dL or greater.
High cholesterol is estimated to burden the U.S. with a cost between
\$18.5 million and \$77 million each year. CVD, a result of having high
cholesterol, is estimated to cost the U.S. \$219 billion dollars.

Prior research has examined long working hours and its association with
blood cholesterol levels but they either studied populations outside of
the U.S. or had very high cut points for working hours. This analysis is
unique in that it studies a random selection of U.S. adults and lowers
the cut points in order to produce results more closely related to the
general U.S. adult population.

\hypertarget{research-question}{%
\subsubsection{Research question}\label{research-question}}

Is greater number of hours worked in the prior week associated with
subsequent high total blood cholesterol among adults aged 18-80 in the
United States, using NHANES 2017-2018 participant data? \textbf{H0:}
There is not an association between greater working hours in the past
week and high total blood cholesterol. \textbf{H1:} There is a
significant positive association between greater working hours in the
past week and high total blood cholesterol.

\hypertarget{methods}{%
\section{Methods}\label{methods}}

\hypertarget{study-design-and-setting}{%
\subsubsection{Study design and
setting}\label{study-design-and-setting}}

The National Center for Health Statistics (NCHS) administered the
National Health and Nutrition Examination Survey (NHANES) during 2017
and 2018. This survey recorded health information, status, and
measurements for the purpose of estimating nationwide disease prevalence
and aiding in health policy development. The sample design was a complex
multi-level process that oversampled and undersampled certain
demographics that later became weighted to represent the whole
noninstitutionalized civilian population of the United States.
Participants were interviewed about their personal demographics, health
status, and behaviors, while measurements were recorded by a mobile
clinic during a standardized physical examination.

\hypertarget{study-population}{%
\subsubsection{Study population}\label{study-population}}

During 2017-2018 the National Health and Nutrition Examination Survey
(NHANES) recruited 9,254 participants. Of these participants, 1,715 were
selected based on their blood tests and questionnaire results. All
selected observations had recorded total blood cholesterol measurements,
at least 30 total hours worked at all jobs during the week prior to
being surveyed, were not currently taking cholesterol medication, and
were at least 18 years of age.

\hypertarget{data-sources-and-measurement}{%
\subsubsection{Data sources and
measurement}\label{data-sources-and-measurement}}

Exposure: Total working hours in the week prior to the survey
administration was obtained from a series of questions. First
participants were asked ``In this part of the survey I will ask you
questions about your work experience. Which of the following were you
doing last week?'' The options to answer this question were as followed:
Working at a job or business; with a job or business but not at work;
looking for work; not working at a job or business; refused; or don't
know. Only those who responded that they worked at a job or business
were considered for this study. A follow-up question was asked to those
who responded as such: ``How many hours did you work last week at all
jobs or businesses?'' Those who answered between 1 and 5 hours were
recorded as ``5.'' Six to 78 hours were recorded as discrete values. No
respondents reported 79 hours, and those who reported 80 or more were
recorded as ``80.'' Refusals to report and ``Don't know'' were also
recorded. The goal of this study was to examine full-time workers, so
observations were dropped from the analysis if they worked less than 30
hours, refused to report, or did not know how many hours they worked
last week.

Outcome: Total blood cholesterol was recorded by a combined effort of
collecting blood samples by the mobile examination clinic and enzymatic
assay methods performed by contracted laboratories. Collection of the
samples occurred immediately prior to questionnaire data was obtained.
After the completion of the laboratory analyses, total blood cholesterol
was recorded as discrete values with units of milligrams per deciliter
(mg/dL) of blood. No parameters were placed to make exclusions based on
these results.

Covariates: Variables included in the study were chosen based on modern
knowledge of risk factors related to high total blood cholesterol, along
with common covariates of past research with similar topics. Two
variables were modified from the NHANES 2017-2018 dataset to consolidate
similar categories. Those who self-reported being Mexican American or
Other Hispanic were collapsed into a single Hispanic category. Also,
having not attended high school was combined with not finishing high
school or obtaining a GED. This group was considered as No High School
Diploma.

\hypertarget{efforts-to-address-bias}{%
\subsubsection{Efforts to address bias}\label{efforts-to-address-bias}}

Those who worked less than full-time in the prior week, defined as less
than 30 hours, were removed from the study to prevent a potential bias
that could come from observations who had not worked at all or worked
very little. Children and adolescents were removed as they may not have
lived long enough to observe direct effects of working hours on total
blood cholesterol. The final model controlled for the effects of the
selected covariates to measure the association more accurately between
prior week working hours and total blood cholesterol.

\hypertarget{statistical-methods}{%
\subsubsection{Statistical methods}\label{statistical-methods}}

Descriptive and analytic statistics were calculated with R 4.1.1 ``Kick
Things'' and utilized the weights included with the NHANES 2017-2018
dataset. No unweighted statistics are listed in Tables 1-3. Total blood
cholesterol was normally distributed amongst the weighted population;
therefore, the mean and standard deviation were included in Table 1. All
other continuous variables were nonnormal and the medians with minimum
and maximum values were reported.

Simple linear regression was used to determine the effects of the
continuous variables on total blood cholesterol for the bivariate
analysis. Pearson's correlation was used to describe the strength of
these effects. Bartlett's test for homogeneity was used to determine the
inclusion of a categorical variable in the bivariate analysis.
Independent t-tests and ANOVA were performed to calculate differences in
means for the included categorical variables. Multiple linear regression
was used to build the adjusted model of determining the effect of prior
week working hours on total blood cholesterol.

\hypertarget{results}{%
\section{Results}\label{results}}

\hypertarget{key-findings}{%
\subsubsection{Key findings}\label{key-findings}}

\begin{itemize}
\tightlist
\item
  37.0\% of the weighted population had clinically high total blood
  cholesterol.
\item
  47.0\% reported working more than 40 hours in the prior week.
\item
  Each hour worked was associated with a decrease of total blood
  cholesterol by 0.28 mg/dL after adjusting for covariates.
\item
  Race and vigorous recreational activity had the largest effect on
  total blood cholesterol in the adjusted model.
\end{itemize}

\hypertarget{study-population-1}{%
\subsubsection{Study population}\label{study-population-1}}

The NHANES 2017-2018 dataset consisted of 9,254 observations that was
reduced to 1,715 observations due to the exclusion criteria of this
study. The final study population was weighted to represent 95,960,477
non-institutionalized adults of the United States.

\hypertarget{demographics}{%
\subsubsection{Demographics}\label{demographics}}

The weighted interquartile range for age was 29 -- 51 years old. A BMI
of 25 or greater was recorded for 72.9\% of the weighted population (SE
= 1.6\%). The prevalence of either college attendance or completion of
an advanced degree was 66.2\% (SE = 1.7\%). Household income at or below
poverty lines was calculated in 9.0\% of the weighted population (SE =
0.7\%).

\hypertarget{total-blood-cholesterol}{%
\subsubsection{Total blood cholesterol}\label{total-blood-cholesterol}}

Total blood cholesterol was normally distributed within the weighted
population and had a mean of 190.0 mg/dL (SE = 1.4). Approximately
37.0\% had a clinically high total blood cholesterol of over 200 mg/dL
(SE = 1.8\%) and 9.9\% had a very high blood cholesterol of 240 mg/dL or
higher (SE = 1.1\%).

\hypertarget{working-hours}{%
\subsubsection{Working hours}\label{working-hours}}

Approximately 33.6\% of the weighted population worked exactly 40 hours
(SE = 1.7\%) and 46.9\% worked more than 40 hours (SE = 1.8\%). The
proportion of those who worked between 30-39 hours was 19.4\% (SE =
1.4\%). Male mean working hours was 46.7 hours (SE = 0.6) whereas the
female mean was 42.8 hours (SE = 0.5).

\hypertarget{bivariate-analysis}{%
\subsubsection{Bivariate analysis}\label{bivariate-analysis}}

Prior week working hours had a weak negative unadjusted correlation with
total blood cholesterol (R = -0.08) and accounted for a decrease in
total blood cholesterol by 0.31 mg/dL for every hour worked (95\% CI =
-0.53 -- -0.09). The strongest positive unadjusted correlation with
total blood cholesterol were age (R = 0.31) and BMI (R = 0.12). An
increase of 1 year in age was associated with an increase of 0.93 mg/dL
(95\% CI = 0.72 -- 1.1) in total blood cholesterol. For BMI, an increase
of 1 unit was associated with an increase of total blood cholesterol by
0.65 mg/dL (95\% CI = 0.32 -- 0.97). Taking part in vigorous
recreational physical activity reduced total blood cholesterol by 11.1
mg/dL (95\% CI = 5.8 -- 16.3). Non-Hispanic Blacks had the lowest mean
total blood cholesterol (mean = 182.5 mg/dL, SE = 2.2) while the Other
and Multi-racial category had the highest (mean = 200.0 mg/dL, SE =
5.3). Higher levels of education generally decreased mean total blood
cholesterol, but these differences were not significant.

\hypertarget{multivariable-analysis}{%
\subsubsection{Multivariable analysis}\label{multivariable-analysis}}

The adjusted model more accurately predicted the variance in total blood
cholesterol (Adjusted R2 = 0.12) compared to the unadjusted model of
prior week working hours (Adjusted R2 = -0.0004), though both were not
strong predictors. Each hour worked was associated with an average
reduction in total blood cholesterol by 0.28 mg/dl (95\% CI = -0.44 --
-0.11) after adjusting for age, BMI, rigorous recreational activity,
income-to-poverty ratio, and race/ethnicity. Race and self-reporting
vigorous recreational physical activity had the largest effects on total
blood cholesterol in the adjusted model.

\hypertarget{references}{%
\section*{References}\label{references}}
\addcontentsline{toc}{section}{References}

\hypertarget{refs}{}
\begin{CSLReferences}{0}{0}
\leavevmode\vadjust pre{\hypertarget{ref-nhlbi2001}{}}%
\CSLLeftMargin{{[}1{]} }
\CSLRightInline{NHLBI. {ATP III At}-{A}-{Glance}: Quick {Desk
Reference}. National Institutes of Health 2001.}

\leavevmode\vadjust pre{\hypertarget{ref-virani2021}{}}%
\CSLLeftMargin{{[}2{]} }
\CSLRightInline{Virani SS, Alonso A, Aparicio HJ, Benjamin EJ,
Bittencourt MS, Callaway CW, et al. Heart {Disease} and {Stroke
Statistics}\textemdash 2021 {Update}. Circulation 2021;143:e254--743.
doi:\href{https://doi.org/10.1161/CIR.0000000000000950}{10.1161/CIR.0000000000000950}.}

\leavevmode\vadjust pre{\hypertarget{ref-cdc2020}{}}%
\CSLLeftMargin{{[}3{]} }
\CSLRightInline{CDC. Knowing {Your Risk}: High {Cholesterol}. Centers
for Disease Control and Prevention 2020.}

\leavevmode\vadjust pre{\hypertarget{ref-ferrara2021}{}}%
\CSLLeftMargin{{[}4{]} }
\CSLRightInline{Ferrara P, Laura DD, Cortesi PA, Mantovani LG. The
economic impact of hypercholesterolemia and mixed dyslipidemia: A
systematic review of cost of illness studies. PLOS ONE 2021;16:e0254631.
doi:\href{https://doi.org/10.1371/journal.pone.0254631}{10.1371/journal.pone.0254631}.}

\leavevmode\vadjust pre{\hypertarget{ref-rivera2020}{}}%
\CSLLeftMargin{{[}5{]} }
\CSLRightInline{Rivera AS, Akanbi M, O'Dwyer LC, McHugh M. Shift work
and long work hours and their association with chronic health
conditions: A systematic review of systematic reviews with
meta-analyses. PLOS ONE 2020;15:e0231037.
doi:\href{https://doi.org/10.1371/journal.pone.0231037}{10.1371/journal.pone.0231037}.}

\leavevmode\vadjust pre{\hypertarget{ref-flsa1938}{}}%
\CSLLeftMargin{{[}6{]} }
\CSLRightInline{Fair {Labor Standards Act} 1938:1060--70.}

\leavevmode\vadjust pre{\hypertarget{ref-lee2016}{}}%
\CSLLeftMargin{{[}7{]} }
\CSLRightInline{Lee D-W, Hong Y-C, Min K-B, Kim T-S, Kim M-S, Kang M-Y.
The effect of long working hours on 10-year risk of coronary heart
disease and stroke in the {Korean} population: The {Korea National
Health} and {Nutrition Examination Survey} ({KNHANES}), 2007 to 2013.
Annals of Occupational and Environmental Medicine 2016;28.
doi:\href{https://doi.org/10.1186/s40557-016-0149-5}{10.1186/s40557-016-0149-5}.}

\leavevmode\vadjust pre{\hypertarget{ref-lemke2017}{}}%
\CSLLeftMargin{{[}8{]} }
\CSLRightInline{Lemke MK, Apostolopoulos Y, Hege A, Wideman L, Sönmez S.
Work, sleep, and cholesterol levels of {U}.{S}. Long-haul truck drivers.
Industrial Health 2017;55:149--61.
doi:\href{https://doi.org/10.2486/indhealth.2016-0127}{10.2486/indhealth.2016-0127}.}

\leavevmode\vadjust pre{\hypertarget{ref-ppaca2010}{}}%
\CSLLeftMargin{{[}9{]} }
\CSLRightInline{The {Patient Protection} and {Affordable Care Act}
2010:119--025.}

\leavevmode\vadjust pre{\hypertarget{ref-artazcoz2009}{}}%
\CSLLeftMargin{{[}10{]} }
\CSLRightInline{Artazcoz L, Cortès I, Escribà-Agüir V, Cascant L,
Villegas R. Understanding the relationship of long working hours with
health status and health-related behaviours. Journal of Epidemiology and
Community Health (1979-) 2009;63:521--7.}

\leavevmode\vadjust pre{\hypertarget{ref-kivimaki2018}{}}%
\CSLLeftMargin{{[}11{]} }
\CSLRightInline{Kivimäki M, Steptoe A. Effects of stress on the
development and progression of cardiovascular disease. Nature Reviews
Cardiology 2018;15:215--30.
doi:\href{https://doi.org/10.1038/nrcardio.2017.189}{10.1038/nrcardio.2017.189}.}

\leavevmode\vadjust pre{\hypertarget{ref-sasaki1999}{}}%
\CSLLeftMargin{{[}12{]} }
\CSLRightInline{Sasaki T, Iwasaki K, Oka T, Hisanaga N. Association of
{Working Hours} with {Biological Indices Related} to the {Cardiovascular
System} among {Engineers} in a {Machinery Manufacturing Company}.
INDUSTRIAL HEALTH 1999;37:457--63.
doi:\href{https://doi.org/10.2486/indhealth.37.457}{10.2486/indhealth.37.457}.}

\leavevmode\vadjust pre{\hypertarget{ref-virtanen2019}{}}%
\CSLLeftMargin{{[}13{]} }
\CSLRightInline{Virtanen M, Magnusson Hansson L, Goldberg M, Zins M,
Stenholm S, Vahtera J, et al. Long working hours, anthropometry, lung
function, blood pressure and blood-based biomarkers: Cross-sectional
findings from the {CONSTANCES} study. Journal of Epidemiology and
Community Health 2019;73:130--5.
doi:\href{https://doi.org/10.1136/jech-2018-210943}{10.1136/jech-2018-210943}.}

\leavevmode\vadjust pre{\hypertarget{ref-kivimaki2015}{}}%
\CSLLeftMargin{{[}14{]} }
\CSLRightInline{Kivimäki M, Jokela M, Nyberg ST, Singh-Manoux A,
Fransson EI, Alfredsson L, et al. Long working hours and risk of
coronary heart disease and stroke: A systematic review and meta-analysis
of published and unpublished data for 603 838 individuals. The Lancet
2015;386:1739--46.
doi:\href{https://doi.org/10.1016/S0140-6736(15)60295-1}{10.1016/S0140-6736(15)60295-1}.}

\leavevmode\vadjust pre{\hypertarget{ref-reynolds2018}{}}%
\CSLLeftMargin{{[}15{]} }
\CSLRightInline{Reynolds AC, Bucks RS, Paterson JL, Ferguson SA, Mori
TA, McArdle N, et al. Working (longer than) 9 to 5: Are there
cardiometabolic health risks for young {Australian} workers who report
longer than 38-h working weeks? International Archives of Occupational
and Environmental Health 2018;91:403--12.
doi:\href{https://doi.org/10.1007/s00420-018-1289-4}{10.1007/s00420-018-1289-4}.}

\end{CSLReferences}


\end{document}
